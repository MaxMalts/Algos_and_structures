1.

Представим каждого сотрудника в виде вершины графа. Соединим эти вершины взвешенными ребрами в соответствии с дружественными отношениями (например, если сотрудник $а$ дружит с $b$ и степень их недовольства при разделении равна $k(a, b)$, то мы проводим ребро $(a, b) : w(a, b) = k(a, b)$). Теперь создадим вершины исток $s$ и сток $t$ и проведем из них ребра ко всем сотрудникам: если степень недовольства сотрудника $a$ попаданием в мат. группу равна $m(a)$, а в прог. группу - $p(a)$, то проводим $(s, a) : w(s, a) = p(a);\; (t, a) : w(t, a) = m(a)$). Теперь ищем минимальный разрез между вершинами $s$ и $t$. Это и будет ответом.

Действительно, если сотрудник $a$ попал в мат. группу, т.е. оказался в одном множестве с $s$, то степень его недовольства по поводу группы оказалась $w(a, t) = m(a)$, и наоборот. При этом если он оказался разделенным с другом $b$, то степень недовольства $a$ и $b$ по поводу друга оказалась $w(a, b) = k(a, b)$. Таким образом, вес каждого разреза соответствует степени недовольства.

Если искать минимальный вес (а значит максимальный поток) с помощью алгоритма Форда-Фалкерсона, то получим $O(\mathit{ans}(2n + m)) = O(\mathit{ans}(n + m))$. Если с помощью Эдмонса-Карпа, то $O\left(n \cdot (m + 2n)^2\right) = O\left(n^3 + n^2 m + nm^2\right)$. Если с помощью Диница, то $O\left(n^2 \cdot (m + 2n)\right) = O\left(n^3 + n^2 m\right)$.