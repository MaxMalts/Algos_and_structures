4.

Создадим вершины исток и сток и соединим их со всеми данными вершинами, ориентировав ребра так, что из истока все ребра выходят, а в сток входят. Ориентируем неориентированные ребра произвольно. и зададим им пропускную способность $1$. Для каждой вершины посчитаем разность количеств входящих в нее ребер и выходящих. Если она отрицательна, то в качестве пропускной способности ребра, исходящего из стока и входящего в текущую вершину, ставим модуль этой разности. Если положительна, то делаем тоже самое, но для ребра, идущего в сток. Остальные пропускные способности делаем $0$. Пускаем максимальный поток из истока в сток. Те ребра, через которые течет поток, инвертируем.

Асимптотики совпадают с асимптотиками из предыдущих задач.