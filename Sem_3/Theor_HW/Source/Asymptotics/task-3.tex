\begin{task}{3}
Сведем к языку $\text{3SAT} = \{\varphi \mid \varphi \text{~--- выполнимая формула в $3$-КНФ}\}$, ставив слову из 3SAT соответствующий граф $G$, аналогично задаче $8~a)$ из семинара. Он будет состоять из треугольников, соединенных по взаимоисключающим вершинам. Пусть у нас есть набор аргументов, при которых функция $\varphi$ истинна. Тогда мы можем выбрать соответствующие независимые вершины в графе $G$, соответствующие этому набору. Но т.к. в каждом треугольнике мы выбрали ровно по одной вершине, то количество выбранных вершин равно $\alpha(G) = \frac{1}{3}|V(G)|$. Также мы можем поставить выбранным независимым вершинам набор аргументов $\varphi$. Таким образом, мы свели 3SAT к нашему языку.
\end{task}