\begin{task}{1}
\begin{enumerate}
    \item[1)] Если убрать требование конечности $Q$, то любой язык разрешим. Действительно, т.к. множество всех слов $\Sigma^*$ счетно, то для каждого слова можно сделать отдельное состояние. На каждом шаге работы машины мы просто будем переходить в состояние, соответствующее слову из текущего состояния с приписанным в конце текущим символом или в состояние $accept$ или $reject$, в зависимости, есть ли слово в алфавите. Головка при этом будет сдвигаться вправо.
    
    \item[2)] Мы не можем убрать требование конечности $\Sigma$, не убрав это требование у $\Gamma$, т.к. $\Sigma \subset \Gamma$.
    
    \item[3)] Если убрать конечность $\Gamma$, то любой язык разрешим, т.к. каждому слову можно сопоставить свой символ из $\Gamma$. Тогда на распознавание слова потребуется всего один шаг.
\end{enumerate}
\end{task}