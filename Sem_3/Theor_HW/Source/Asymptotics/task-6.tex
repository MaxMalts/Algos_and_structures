\begin{task}{6}
Пусть алфавит $\mathcal{A} = \{0, 1\}$, язык $\mathcal{X} \subset \mathcal{A}^*$ NP-полный. Возьмем $A = \{0x \mid x \in \mathcal{X}\}$, $B$~--- множество всех слов, начинающихся на $0$ ($B = \{0a \mid a \in \mathcal{A}^*\}$), $C = B \cup \{1x \mid x \in \mathcal{X}\}$. Таким образом, $A \subset B \subset C$, при этом очевидно, что $B \in \text{P}$, а т.к. $\mathcal{X}$ NP-полный, то $A$ и $C$ NP-полны, 
\end{task}